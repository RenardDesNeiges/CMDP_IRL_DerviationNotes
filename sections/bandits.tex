In order to work towards a better understanding of the IRL problems at hand we first restrict ourselves to the study of a single-state MDP, which is equivalent to a \textit{$K$-Armed Bandit Problem}.

\subsection{A few observations about Bandit problems}

The "\textit{bandit}" (a.k.a single state MDP has a few interesting properties that make the analysis of optimization algorithms on it simpler). In the bandit setting the policy $\bm{\pi} \in \Delta_A$ becomes simply a distribution on the action-set rather than a function form the states to distributions on the actions-set, we can thus represent $\bm{\pi}$ as a vector in $\mathbb{R}^n$

\begin{observation}
    (\textbf{Policy-Occupancy Measure Equality}) In the specific case of the bandit setting, we have that:
    \begin{align*}
        \bm{\mu}^\pi = \bm{\pi}.
    \end{align*}
    This can easily be verified as follows:
    \begin{align*}
        \mu^\pi(a) &= (1-\gamma) \lim_{T \rightarrow + \infty} \sum_{t=0}^T \gamma^t \mathbb{P}_\nu^\pi(a_t=a) 
        \\
        &= (1-\gamma) \lim_{T \rightarrow + \infty} \sum_{t=0}^T \gamma^t \pi(a_t=a) \\
        &= \pi(a).
    \end{align*}
\end{observation}

\begin{observation}
    (\textbf{Objective Function in the Bandit Setting}) In the specific case of the bandit setting, we have that the objective function reduces to:
    \begin{align*}
        J^\nu(\pi) &= \langle \bm{r},\bm{\mu}^\pi \rangle -\tilde{\Omega}(\bm{\pi})\\
        &= \langle \bm{r}, \bm{\pi} \rangle - \Omega(\bm{\pi}).
    \end{align*}
\end{observation}

\begin{observation}
    In the bandit setting, the Bellman Flow constraints are always trivially satisfied (the probability that the agent ends up in state $s$ at time $t+1$ is always $1$). 
\end{observation}

Acceptable policies are defined by the feasible set (as for CRL), which is here given by:
\begin{align*}
    \mathcal{F} := \big\lbrace \bm{\pi} \in \Delta_A, \Psi \bm{\pi} \leq \bm{b} \big\rbrace,
\end{align*}
where $\Psi$ is the cost matrix and $\bm{b}$ the constraint vector (see section \ref{sec:cmdps} for details).

\begin{definition}
    (\textbf{Constrained Bandit Solution Map}) we define the constrained bandit solution map $\text{CB}:\mathcal{R} \rightarrow \Pi$ as:
    \begin{align*}
        \text{CB}(\bm{r}) = &\arg \max_{\pi\in \mathcal{F}}  J(\bm{\pi},\bm{r})
    \end{align*}
    (note that here $\mathcal{F} \subseteq \Pi = \mathcal{M} = \Delta_A$).
\end{definition}

% \subsection{A first stab at proving convergence of IRL : the Unconstrained Inverse Bandit Problem} \label{sec:unconstrained_bandit_problem}
% Consider a MDP $\textit{M}$ with a unique state $s$. We rewrite the IRL problem as stated in proposition \ref{proposition:minmaxIRL} for that single-state problem as:

% \begin{equation}
%     \label{eq:minmaxBanditIRL}
%     \begin{aligned}
%         \min_{\bm{r}\in\mathcal{R}}\max_{\bm{\pi}\in\Delta_{A}} \bm{r}^T(\bm{\pi}-\bm{\pi}^E)-\Omega(\bm{\pi}),
%     \end{aligned}
% \end{equation}

% in order to more specifically define our problem let us pick a reward class $\mathcal{R}$ as well as a convex regularizer $\Omega:\Pi \rightarrow \mathbb{R}$. Specifically we choose our reward class to be the $L_1$ ball centered on the origin:
% \begin{align*}
%     \mathcal{R}_{L_1} := \lbrace \bm{r} \in \mathbb{R}^n ; \|r\|_1 \leq 1 \rbrace,
% \end{align*}
% and our regularizer $\Omega$ to be the negative Rényi Entropy with parameter $\alpha=2$:
% \begin{align*}
%     \Omega(\bm{\pi}) = -\beta H_{\alpha=2}(\bm{\pi})  &= -\frac{\beta}{1-\alpha} \log  \sum_{a \in A} \pi_a^\alpha,\\
%     &= \beta \log  \sum_{a \in A} \pi_a^2,
% \end{align*}
% where $\beta>0$, which is a strictly-convex function w.r.t. $\pi$. Rewriting (\ref{eq:minmaxBanditIRL}) we get:

% \begin{equation}
%     \label{eq:minmaxBanditIRLexplicit}
%     \begin{aligned}
%         \min_{\bm{r}\in\mathcal{R}_{L_1}}\max_{\bm{\pi}\in\Delta_{A}} \bm{r}^T(\bm{\pi}-\bm{\pi}^E) + \beta H_2(\bm{\pi}),
%     \end{aligned}
% \end{equation}
% or equivalently:
% \begin{align*}
%     \begin{aligned}
%         \min_{\bm{r}} \max_{\bm{\pi}} ~ & \bm{r}^T(\bm{\pi}-\bm{\pi}^E) + \beta H_2(\bm{\pi}), \\
%         \text{s.t. }& \bm{1}^T \pi = 1, \\
%          & \| \bm{r} \|_1 \leq 1.
%     \end{aligned}
% \end{align*}
% For convenience, let us denote our objective function as $\text{obj}(\bm{r},\bm{\pi}) = \bm{r}^T(\bm{\pi}-\bm{\pi}^E) + \beta H_2(\bm{\pi})$. 
% \begin{observation}
%     \label{obs:properties_unconstrained_inverse_bandit}
%     (\textbf{Properties of the unconstrained bandit problem}) Observe that:
%     \begin{enumerate}
%         \item The domain of the variable $\bm{\pi}$, $\Delta_A$ (the probability simplex on the discrete set $A$) is convex and compact.
%         \item The domain of the variable $\bm{r}$, $\mathcal{R}_{L_1}$ (the L1 ball) is convex and compact.
%         \item $\text{obj}$ is {linear} (and thus, non-strictly, convex) w.r.t the decision variable $\bm{r}$, it is also Lipschitz and smooth (has  Lipschitz Gradients) w.r.t. to both decision variables  $\bm{r}$ and $\bm{\pi}$ on their respective domains $\mathcal{R}_{L_1}$ and $\Delta_A$.
%         \item $\text{obj}$ is concave w.r.t the decision variable $\bm{\pi}$ (since it has a linear term and Rényi's entropy is concave), it is also Lipschitz and smooth (the $\Omega$ function that we chose has $2\beta$-Lipschitz gradients) w.r.t. to both decision variables  $\bm{r}$ and $\bm{\pi}$ on their respective domains $\mathcal{R}_{L_1}$ and $\Delta_A$.
%     \end{enumerate}
% \end{observation}

% \begin{proposition}
%     (\textbf{Convergence of averaged-GDA on the Unconstrained Inverse Bandit Problem})
%     Under the observations made in obs. \ref{obs:properties_unconstrained_inverse_bandit} and assuming \textit{direct access to the gradients} in $\bm{r}$ and in $\bm{\pi}$, directly applying projected gradient descent-ascent updates (as in \ref{alg:projGDA}) gives an algorithm for which, after $K$ steps, the duality gap of the averaged step $\bar{\bm{z}}=(\bar{\bm{r}}_K,\bar{\bm{\pi}}_K) $ is bounded by:
%     \begin{align*}            
%         \text{Duality Gap}(\bar{\bm{r}}_K,\bar{\bm{\pi}}_K) &=  \bm{r}^T(\bm{\pi}_K-\bm{\pi}^E) - \beta (H_2(\bm{\pi}_K) -  H_2(\bm{\pi}^E) ) \\ &\leq \max \Biggl(\frac{16 \beta\sqrt{n}}{\sqrt{K}},\frac{16 \sqrt{n} }{\sqrt{K}} \Biggr) = O(1/\sqrt{K}),
%     \end{align*}

% \end{proposition}
% \begin{proof}
%     By direct computation. Observe that:
%     \begin{enumerate}
%         \item the max diameter of the constraint sets $\Delta_A$ and $\mathcal{R}_{L_1}$ is that of the $L_1$ ball, and that $\text{diam}(\mathcal{R}_{L_1}) = 2$,
%         \item the function $-\beta H_2(\bm{\pi})$ has $2 \beta \sqrt{n}$-Lipschitz gradients w.r.t $\bm{\pi}$ and $2 \sqrt{n}$-Lipschitz gradients w.r.t $\bm{r}$.
%     \end{enumerate}
%     The results follows from the analysis of the projected GDA algorithm (see proposition \ref{prop:proj_gda_convex_concave}).
% \end{proof}

% \begin{proposition}
%     (\textbf{Convergence of averaged-EG on the Unconstrained Inverse Bandit Problem})
%     Under the observations made in obs. \ref{obs:properties_unconstrained_inverse_bandit} and assuming \textit{direct access to the gradients} in $\bm{r}$ and in $\bm{\pi}$, directly applying projected expected gradient updates (as in \ref{alg:projEG}) gives an algorithm for which, after $K$ steps, the duality gap of the averaged step $\bar{\bm{z}}=(\bar{\bm{r}}_K,\bar{\bm{\pi}}_K) $ is bounded by:
%     \[ 
%         \text{Duality Gap}(\bar{\bm{z}}_K) \leq \frac{16 \sqrt{n}}{K} = O(1/{K}),
%     \]
% \end{proposition}
% \begin{proof}
%     By direct computation. Observe that:
%     \begin{enumerate}
%         \item the max diameter of the constraint sets $\Delta_A$ and $\mathcal{R}_{L_1}$ is that of the $L_1$ ball, and that $\text{diam}(\mathcal{R}_{L_1}) = 2$,
%         \item the function $-\beta H_2(\bm{\pi})$ is $2 \beta \sqrt{n}$-smooth.
%     \end{enumerate}
%     The results follows from the analysis of the projected EG algorithm (see proposition \ref{prop:proj_eg_smooth_convex_concave}).
% \end{proof}


\subsection{Constrained Inverse bandit (CIB)} \label{sec:constrained_bandit_problem}

\begin{definition}
    \label{def:cib_sol_map}
    (\textbf{CIRL Solution Map})
    The \textbf{goal of CIB} is to find some mapping $\text{CIRL}_\pi : \Pi \rightarrow \mathcal{R}$.
    \begin{align*}
        (\text{CB}_\pi \circ \text{CIB}_\pi)(\bm{\pi}^E) = \bm{\pi}^E.
    \end{align*}
\end{definition}

We consider the following constraint optimization problem:
\begin{equation}
    \label{eq:minmaxConstrainedBanditIRL}
    \begin{aligned}
        \min_{\bm{r}\in\mathcal{R}}\max_{\bm{\pi}\in\mathcal{F}} \langle \bm{r}, \bm{\pi}-\bm{\pi}^E \rangle-\Omega(\bm{\pi}),
    \end{aligned}
\end{equation}
which provides a way of evaluating \ref{def:cib_sol_map} (this can be shown to be true by proposition \ref{proposition:minmaxIRL}). The Lagrangian dual of \ref{eq:minmaxConstrainedBanditIRL}, found by relaxing the feasibility constraints is given by:
\begin{equation}
    \label{eq:dualConstrainedBanditIRL}
    \begin{aligned}
        \min_{\bm{r}\in\mathcal{R},\bm{\lambda} \geq 0}\max_{\bm{\pi}\in\Delta_{A}} f_\text{CIB}(\bm{r},\bm{\lambda},\bm{\pi}).
    \end{aligned}
\end{equation}
where $f_\text{CIB}(\bm{r},\bm{\lambda},\bm{\pi}) = \langle \bm{r}, \bm{\pi}-\bm{\pi}^E \rangle -\Omega(\bm{\pi}) + \langle \bm{\lambda}, \bm{b}-\Psi \bm{\pi}\rangle$ is our objective function.
\begin{proposition}
    \label{prop:bandit_strong_duality}
    (\textbf{Strong Duality}) Assuming that $\exists \bm{\pi}^E\in\mathcal{F}$, and that $\text{obj}$ is strictly convex. Then dual optimum is attained for some $\bm{\lambda}^* \geq 0$ and problem \ref{eq:minmaxConstrainedBanditIRL} is equivalent to an unconstrained bandit problem with reward $\bm{r}-\Psi \bm{\lambda}^*$. I.e. 
    \begin{align*}
        \text{CRL}_\pi(\bm{r}) = \text{RL}_\pi(\bm{r} - \bm{\Psi} \bm{\lambda}^*).
    \end{align*}
\end{proposition}
\begin{proof}
    Using generic Lagrangian Duality theory, we make two observations:
    \begin{enumerate}
        \item observe that the problem: $\max_{\bm{\pi} \in \mathcal{F}} \langle \bm{r}, \bm{\pi} \rangle - \Omega(\bm{\pi})$, is a \textbf{strictly convex optimization problem},
        \item the primal optimum $\bm{\pi}^*$ is finite as the feasible set $\mathcal{F}$ is bounded and the objective is upper-bounded (since $\Omega$ is strictly convex).
    \end{enumerate}
    From Slater's condition it follows that strong duality holds.
\end{proof}
\noindent
From proposition \ref{prop:bandit_strong_duality} we know that solving problem \ref{eq:dualConstrainedBanditIRL} is equivalent to solving \ref{eq:minmaxConstrainedBanditIRL}. We will study convergence of extra-gradient-based saddle-point algorithms on \ref{eq:dualConstrainedBanditIRL}. \\%Taking the same setup as in section \ref{sec:unconstrained_bandit_problem} ($\Omega(\bm{\pi}) = H_2(\bm{\pi})$, $\mathcal{R}=\mathcal{R}_{L_1})$, we can follow a very similar analysis. \\


% \begin{proposition}
%     (\textbf{Convergence of averaged-EG on the Constrained Inverse Bandit Problem})
%     Under the observations made in obs. \ref{obs:properties_constrained_inverse_bandit} and assuming \textit{direct access to the gradients} in $\bm{r}$ and in $\bm{\pi}$, directly applying projected expected gradient updates (as in \ref{alg:projEG}) gives an algorithm for which, after $K$ steps, the duality gap of the averaged step $\bar{\bm{z}}=(\bar{\bm{r}}_K,\bar{\bm{\pi}}_K) $ is bounded by:
%     \[ 
%         \text{Duality Gap}(\bar{\bm{z}}_K) = O(1/{K}),
%     \]
% \end{proposition}
% \begin{proof}
%     By direct computation. Observe that:
%     \begin{enumerate}
%         \item the max diameter cannot be bounded by the constraint set, but one can just pick $\|\bm{z}_0-\bm{z}^*\|$,
%         \item the constraints $\bm{r} \in \mathcal{R}_{L_1}$, $\bm{\pi} \in \Delta_A$ and $\bm{\lambda} \geq 0$ are convex,
%         \item the function $-\beta H_2(\bm{\pi})$ is $2 \beta \sqrt{n}$-smooth.
%     \end{enumerate}
%     The results follows from the analysis of the projected EG algorithm (see proposition \ref{prop:proj_eg_smooth_convex_concave}).
% \end{proof}
