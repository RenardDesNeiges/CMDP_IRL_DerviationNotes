We will now consider the inverse problem to the MDP and CMDP problems that we discussed before, \textit{Inverse Reinforcement Learning} (IRL). In the IRL setting (\textit{on a known MDP}) we are given an \textit{MDP} or a \textit{CMDP} without its reward:
\begin{align*}
    \textit{CMDP} \setminus \bm{r} = (S,A,P,\bm{\nu},\Psi,\bm{b},\gamma),
\end{align*}
as well as a dataset $\mathcal{D}$ of expert example in the form of trajectories:
\begin{align*}
    \mathcal{D} = \{  \tau_1,\tau_2,...,\tau_N\},
\end{align*}
produced by some expert policy $\pi^E$. Assuming the dataset is large enough (which we will do as first), this is equivalent to getting an approximation of the expert's occupancy measure $\bm{\mu}^E$. In the IRL setting we also generally restrict rewards to a reward class $\mathcal{R}$ (which is some arbitrary convex set in $\mathbb{R}^{nm}$).
\begin{assumption}
    \label{assumption:realizability}
    (\textbf{Realizability})
    We assume that the expert policy is optimal w.r.t. some reward $\bm{r}^E \in \mathcal{R}$, i.e. 
    \begin{align*}
        \bm{\mu}^E = \text{OPT}_{\mu}(\bm{r}^E).
    \end{align*}
\end{assumption}
\noindent
The goal of the IRL problem is to recover a reward $\hat{\bm{r}}$ s.t. the expert $\bm{\mu}^E = \text{OPT}_{\mu,\Psi}(\hat{\bm{r}})$ (the expert is optimal for that reward).

\begin{proposition}
    \label{proposition:minmaxIRL}
    (\textbf{Min-Max Program to solve IRL})
    If assumption \ref{assumption:realizability} is true, then the rewards optimizing the program:
    \begin{equation}
        \label{eq:minmaxIRL}
        \begin{aligned}
            \min_{\bm{r}\in\mathcal{R}}\max_{\bm{\mu}\in\mathcal{F}} \bm{r}^T(\bm{\mu}-\bm{\mu}^E)-\tilde{\Omega}(\bm{\mu}),
        \end{aligned}
    \end{equation}
    are exactly the rewards in $\mathcal{R}$ for which $\bm{\mu}^E$ is optimal.
\end{proposition}
\begin{proof}
    (Of proposition \ref{proposition:minmaxIRL}) We can rewrite an equivalent problem to (\ref{eq:minmaxIRL}) as follows:
    \begin{align*}
        \min_{\bm{r}\in\mathcal{R}}\max_{\bm{\mu}\in\mathcal{F}} \bm{r}^T\bm{\mu}-\tilde{\Omega}(\bm{\mu}) - \bm{r}^T \bm{\mu}^E + f(\bm{\mu}^E),
    \end{align*}
    where the addition of the $f(\bm{\mu}^E)$ term has no influence on the optimizers, as it is a function of neither of the decision variables. For convenience we write:
    \begin{align*}
        \bm{r}^T\bm{\mu}-\tilde{\Omega}(\bm{\mu}) - \bm{r}^T \bm{\mu}^E + f(\bm{\mu}^E) = L(\bm{r},\bm{\mu}).
    \end{align*}
    For any fixed $\bm{r}\in\mathcal{R}$ it holds that 
    \begin{align*}
        \arg\max_{\bm{\mu}\in\mathcal{F}} \bm{r}^T\bm{\mu}-\tilde{\Omega}(\bm{\mu}) - \overbrace{\bm{r}^T \bm{\mu}^E + f(\bm{\mu}^E)}^\text{constant w.r.t $\bm{\mu}$} = \text{OPT}_\mu(\bm{r}), ~\bm{r}\in \mathcal{R},
    \end{align*}
    i.e. the optimal $\bm{\mu}$ for any fixed $\bm{r} \in \mathcal{R}$ gives the optimal policy. Moreover, for any fixed $\bm{r} \in \mathcal{R}$, we get that:
    \begin{align*}
        \max_{\bm{\mu}\in\mathcal{F}} L(\bm{r},\bm{\mu}).\geq 0, ~\bm{r}\in \mathcal{R},
    \end{align*}
    since:
    \begin{enumerate}
        \item either $\mu^E$ maximizes the value $\bm{r}^T \bm{\mu}^E - f(\bm{\mu}^E)$, in which case, we the optimizer can set $\bm{\mu} = \bm{\mu}^E$ and get $ L(\bm{r},\bm{\mu})=0$,
        \item or  $\mu^E$ does not maximize the value $\bm{r}^T \bm{\mu}^E - f(\bm{\mu}^E)$, in  which case we get $ L(\bm{r},\bm{\mu})>0$.
    \end{enumerate}
    Observe that the lower bound ($ L(\bm{r},\bm{\mu})=0$) is only achieved if $\bm{r}^T \bm{\mu}^E - f(\bm{\mu}^E)=\bm{r}^T \bm{\mu} - \tilde{\Omega}(\bm{\mu})$, i.e. if $\bm{\mu}^E \in \arg \max_{\bm{\mu} \in \mathcal{F}} \bm{r}^T \bm{\mu}^E + f(\bm{\mu}^E)$ which is equivalent to $\bm{\mu}^E \in  \text{OPT}_\mu(\bm{r})$. I.e. the optimal reward $\bm{r}^*$ of the min-max problem is such that the expert $\bm{\mu}^E$ is optimal w.r.t $\bm{r}^*$.
    
\end{proof}

% \begin{proposition}
%     (\textbf{Min-Max Program to solve IRL}) 
% \end{proposition}