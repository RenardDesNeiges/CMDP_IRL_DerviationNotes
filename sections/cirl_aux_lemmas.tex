\begin{assumption}[Reward gradients are bounded]
    \label{assumption:bounded_reward_grads}
    We require that $\forall \bm{w} \in \text{dom}(\bm{w})$ the following is true:
    \[ \|\frac{\partial \bm{r}_w}{\partial w}\|_2 \leq B_w, \]
    this is trivially satisfied when consider linear reward classes, specifically we have $\bm{r}_w = \Phi \bm{w}$ and thus:
    \begin{align*}
        \frac{\partial \bm{r}_w}{\partial w}
        = 
        \Phi,
    \end{align*}
    and thus $B_w=\|\Phi\|$ where  $\|\cdot\|$ denotes the spectral norm.
\end{assumption}

\begin{claim}
    \label{claim:square_2_norm_of_policy_bounds_infinity_log_policy_norm}
    For any two policies $\bm{\pi}, \bar{\bm{\pi}} \in \Pi$ we have that:
    \begin{align*}
        \frac{1}{4} \| \bm{\pi} - \bar{\bm{\pi}} \|_2^2
        \leq
        n \cdot \| \log \bm{\pi} - \log \bar{\bm{\pi}} \|_\infty,
    \end{align*}
    where $\log$ are vectorized and $n=|S|$ is the number of states.
    \begin{proof}
        \begin{align*}
            \frac{1}{4} \| \bm{\pi} - \bar{\bm{\pi}} \|_2^2
            & \leq
            n \frac{1}{4} \| \bm{\pi} - \bar{\bm{\pi}} \|_1^2 
            && \|\cdot\|_2 \leq \|\cdot\|_1 \\
            & =
            n \frac{1}{4} \norm{
                    \begin{bmatrix}
                        \bm{\pi} (\cdot|s_1) \\ \vdots \\
                        \bm{\pi} (\cdot|s_n)
                    \end{bmatrix} -
                    \begin{bmatrix}
                        \bar{\bm{\pi}} (\cdot|s_1) \\ \vdots \\
                        \bar{\bm{\pi}} (\cdot|s_n)
                    \end{bmatrix}
                }_1^2 
            && \text{isolate distribs.} \\
            & = \frac{1}{4} \Big( \sum_{s\in S} \| \bm{\pi} (\cdot|s) 
            - \bar{\bm{\pi}} (\cdot|s) \|_1 \Big)^2
            && \text{def of $\|\cdot\|_1$} \\
            & = \sum_{s\in S} \frac{1}{2} \| \bm{\pi} (\cdot|s) 
            - \bar{\bm{\pi}} (\cdot|s) \|_1^2 
            && (a+b)^2 \leq 2(a^2+b^2) \\
            & \leq \sum_{s\in S} D_\text{KL}( \bm{\pi} (\cdot|s) 
            || \bar{\bm{\pi}} (\cdot|s) )
            && \frac{1}{2} \|\bm{p}_1 - \bm{p}_2 \|_1^2 \leq D(\bm{p}_1||\bm{p}_2) \\
            & = \sum_{s\in S} | \bm{\pi}(\cdot|s)^T (  \log \bm{\pi}(\cdot|s) - \log \bar{\bm{\pi}}(\cdot|s) )  |
            && \text{def. of KL}\\
            & \leq \sum_{s\in S} 
            \overbrace{\| \bm{\pi} (\cdot|s) \|_1 }^{=1}
            \cdot \|  \log \bm{\pi} (\cdot|s) - \log \bar{\bm{\pi}} (\cdot|s) \|_\infty  
            && \text{Hödler's ineq.}\\
            & \leq \sum_{s\in S} 
             \|  \log \bm{\pi} (\cdot|s) - \log \bar{\bm{\pi}} (\cdot|s) \|_\infty  
            && \\
            & \leq 
            n \cdot
             \|  \log \bm{\pi} - \log \bar{\bm{\pi}} \|_\infty  
            && \\
        \end{align*}
        The proof hinges on the inequality:
        \[ \frac{1}{2} \|\bm{p}_1 - \bm{p}_2 \|_1^2 \leq D(\bm{p}_1||\bm{p}_2), \]
        which can be found in the Theorem $11.6$ of \cite{Cover2006}.
    \end{proof}
\end{claim}

\begin{claim}
    \label{claim:infinity_log_exp}
    For any two vectors $v_1,v_2 \in \mathbb{R}^n$ we have that:
    \begin{align*}
        |\log \| \exp v_1 \|_1 -\log \| \exp v_2 \|_1| \leq \|v_1-v_2\|_\infty,
    \end{align*}
    where $\exp \cdot$ is applied element-wise.
    \begin{proof}
        We start by using that the mean-value theorem states that there exists a vector $v_c$ which is a convex combination of $v_1$ and $v_2$ s.t. the following equality is verified:
        \[
            | \log \bigl( \| \exp v_1 \|_1 \bigr) - \log \bigl( \| \exp v_2 \|_1 \bigr)  | 
            =  
            | \big\langle 
            v_1 - v_2,
            \nabla_v \log (\exp v) |_{v = v_c}
            \big\rangle |,
        \]
        and then it's just a matter of bounding the right terms:
        \begin{align*}
            &| \log \bigl( \| \exp v_1 \|_1 \bigr) - \log \bigl( \| \exp v_2 \|_1 \bigr)  | \\
            =  &
            | \big\langle 
            v_1 - v_2,
            \nabla_v \log (\exp v) |_{v = v_c}
            \big\rangle | && \text{by the mean value theorem}
            \\
            \leq  & 
            \| v_1 - v_2 \|_\infty \| \nabla_v \log (\exp v) |_{v = v_c} \|_1
            && \text{Hodler's inequality} 
            \\
            =  & 
            \| v_1 - v_2 \|_\infty
            &&
            \| \nabla_v \log (\exp v) |_{v = v_c} \|_1 =
            \Bigg\|\frac{\exp(v_c)}{\|v_c\|_1}\Bigg\|_1 = 1.
        \end{align*}
    \end{proof}
\end{claim}

\begin{claim}[Bounded gradients of the objective function with respect to the Lagrande and reward parameters]
    \label{claim:f_lipschitz_wrt_rew_params}
    For any Shannon regularized CMDP, the following holds:
    \begin{align*}
        \|\nabla_{\bm{w}} f(\bm{w},\bm{\lambda},\bm{\theta})\|_2 \leq 2B_{\bm{w}}\\
        \|\nabla_{\bm{\lambda}} f(\bm{w},\bm{\lambda},\bm{\theta})\|_2 \leq \|\Psi\|.
    \end{align*}
    \begin{proof}
        This is just a matter of calculating the gradients explicitly and using the bounds that we know are true:
        \begin{align*}
            \|\nabla_{\bm{w}} f(\bm{w},\bm{\lambda},\bm{\theta})\|_2
            &=
            \Big\|
                \nabla_{\bm{w}} \Bigl(
                    \langle \bm{r}, \bm{\mu}-\bm{\mu}^E \rangle
                    \\ &
                    -(\tilde{\Omega}(\bm{\mu})-\tilde{\Omega}(\bm{\mu}^E)) + \langle \bm{\lambda}, \bm{b}-\Psi^T \bm{\mu} \rangle
                \Bigr)
                \Big\|_2\\
            &=
            \norm{
                (\bm{\mu}-\bm{\mu}^E)
                \frac{\partial \bm{r}}{\partial \bm{w}}
            }_2\\
            &\leq
            \| (\bm{\mu}-\bm{\mu}^E)\|_2 \cdot
            \| \frac{\partial \bm{r}}{\partial \bm{w}}\|_2
            && \text{triangle ineq}
            \\&\leq 
            \| (\bm{\mu}-\bm{\mu}^E)\|_2 \cdot
            B_w && \text{assumption \ref{assumption:bounded_reward_grads}}
            \\
            &\leq
            2\sup_{\bm{\mu}}  \| \bm{\mu} \|_2 \cdot
            B_w = 2 B_w. && \mu \in \Delta_{A\times S}
            \\
        \end{align*}
        \begin{align*}
            \|\nabla_{\bm{\lambda}} f(\bm{w},\bm{\lambda},\bm{\theta})\|_2
            &=
            \Big\|
                \nabla_{\bm{\lambda}} \Bigl(
                    \langle \bm{r}, \bm{\mu}-\bm{\mu}^E \rangle
                    \\ &
                    -(\tilde{\Omega}(\bm{\mu})-\tilde{\Omega}(\bm{\mu}^E)) + \langle \bm{\lambda}, \bm{b}-\Psi^T \bm{\mu} \rangle
                \Bigr)
                \Big\|_2\\
            &=
            \norm{
                \Psi^T \bm{\mu}
            }_2\\
            &\leq
            \norm{
                \Psi
            }.
            && \text{spectral norm def.}
        \end{align*}
    \end{proof}
\end{claim}

\begin{lemma}[Optimal $Q$-values are Lipschitz in reward and Lagrange parameters]
    \label{lemma:optQ_is_lipschitz_wrt_w}
    Consider the optimal $Q$ values induced by two different rewards in a constrained, Shannon-regularized Markov decision process, we have that, assuming assumption \ref{assumption:bounded_reward_grads} is satisfied, for any $\bm{w}_1,\bm{w_2} \in \text{dom}(\bm{w})$ and any $\bm{\lambda}_1,\bm{\lambda}_2\in\mathcal{B}$, the inequality below holds:
    \begin{align*}
        \| \bm{Q}^{*}_{\tilde{\bm{r_1}}} - \bm{Q}^{*}_{\tilde{\bm{r_2}}} \|_\infty  
        \leq \frac{B_w}{1 - \gamma} \| \bm{w}_1 - \bm{w}_2 \|_2 + \frac{\|\Psi\|}{1 - \gamma} \| \bm{\lambda}_1 - \bm{\lambda}_2 \|_2.
    \end{align*}
    \begin{proof}
        We follow a derivation very close to the one of Lemma 5.3 in \cite{Zeng2022}. We show lipschitzness by bounding the gradient with respect to parameters $\bm{\lambda}$ and $\bm{w}$, since the derivation is the same for both parameters, we write $\nabla_\lozenge$ for the gradient with respect to either $\bm{\lambda}$ or $\bm{w}$. We start from the definition of the optimal $Q$ value associated with diminished reward $\tilde{\bm{r}}$ for some $(s,a)\in S \times A$s state action pair:
        \begin{align*}
            \nabla_\lozenge Q_{\tilde{\bm{r}}}^*(s_0,a_0) 
            &= \nabla_\lozenge \Big[
            \tilde{r}_{w,\lambda}(s_0,a_0)
            +
            \gamma \mathbb{E}_{s_1 \sim P^{\pi}(s_0,a_0)} \big[
                V_{\tilde{\bm{r}}}^*(s) (s_1)
            \big]
            \Big] \\
            &\stackrel{(i)}{=} \nabla_\lozenge 
            \tilde{r}_{w,\lambda}(s_0,a_0)
            +
            \gamma \mathbb{E}_{s_1 \sim P^{\pi}(s_0,a_0)} 
            \Big[
                \nabla_\lozenge 
                \log \big\|
                \exp
                \bm{Q}_{\tilde{\bm{r}}}^*(s_1,\cdot)
                \big\|_1
            \Big]\\
            &= \nabla_\lozenge 
            \tilde{r}_{w,\lambda}(s_0,a_0)
            +
            \gamma \mathbb{E}_{s_1 \sim P^{\pi}(s_0,a_0)} 
            \Bigg[
                \sum_a \frac{Q_{\tilde{\bm{r}}}^*
                }{\sum_{a'} Q_{\tilde{\bm{r}}}^*(s_1,a') }
                \nabla_\lozenge 
                Q_{\tilde{\bm{r}}}^*(s_1,a)
            \Bigg]\\
            &= \nabla_\lozenge 
            \tilde{r}_{w,\lambda}(s_0,a_0)
            +
            \gamma \mathbb{E}_{s_1 \sim P^{\pi}(s_0,a_0)} 
            \Big[
                \sum_a 
                \tilde{\pi}(a|s_1)
                \nabla_\lozenge 
                Q_{\tilde{\bm{r}}}^*(s_1,a)
            \Big]\\
            &= \nabla_\lozenge 
            \tilde{r}_{w,\lambda}(s_0,a_0)
            +
            \gamma \mathbb{E}_{s_1 \sim P^{\pi}(s_0,a_0)~,a_1 \sim \tilde{\pi}} 
            \Big[
                \nabla_\lozenge 
                Q_{\tilde{\bm{r}}}^*(s_1,a_1)
            \Big]\\
            &\stackrel{(ii)}{=} 
            \mathbb{E}_{\tau \sim \tilde{\pi}} 
            \Big[
                \sum_{t=0}^{+\infty} \gamma^t 
                \nabla_\lozenge \tilde{\bm{r}}_{w,\lambda}^{(t)}
            \Big].
        \end{align*}
        We use $(ii)$ that the $V$ value is the average across $Q$ values under policy distribution and simplify until we $(ii)$ identify a recursion. Since we only care about gradient norms we have that:
        \begin{align*}
            \big\| \nabla_\lozenge Q_{\tilde{\bm{r}}}^*(s,a)  \big\|_2 
            &= 
            \norm{\nabla_\lozenge 
            \mathbb{E}_{\tau \sim \tilde{\pi}} 
            \Bigg[
                \sum_{t=0}^{+\infty} \gamma^t 
                \nabla_\lozenge \tilde{\bm{r}}_{w,\lambda}^{(t)}
                \Bigg| s_0=s,a_0=a
            \Bigg]}_2 
            && \\
            &= 
            \mathbb{E}_{\tau \sim \tilde{\pi}}  
            \Bigg[ 
                \sum_{t=0}^{+\infty} \gamma^t 
                \overbrace{\| \nabla_\lozenge \tilde{\bm{r}}_{w,\lambda}^{(t)} \|_2}^{\leq B_\lozenge}
                \Bigg| s_0=s,a_0=a
            \Bigg]
            && \text{Jensen's ineq.} \\
            &\leq
            \mathbb{E}_{\tau \sim \tilde{\pi}}  
            \Bigg[ 
                \sum_{t=0}^{+\infty} \gamma^t 
                B_\lozenge
                \Bigg| s_0=s,a_0=a
            \Bigg]
            && \text{Bouded $\tilde{r}$ gradients.} \\
            &=
            \frac{B_\lozenge}{1 - \gamma}
            && \\
        \end{align*} 
        Now using assumption \ref{assumption:bounded_reward_grads} together with the discounted reward we clearly see that the diminished reward $\tilde{\bm{r}} = \bm{r}_w  - \Psi \bm{\lambda}$ has bounded gradients in $\bm{\lambda}$ and $\bm{w}$:
        \begin{align*}
            \norm{
                \nabla_w \tilde{\bm{r}}
            }= 
            \norm{
                \nabla_w {\bm{r}}
            } \leq B_w
            && \text{assumption \ref{assumption:bounded_reward_grads}}\\
            \norm{
                \nabla_\lambda \tilde{\bm{r}}
            } \leq 
            \| \Psi \|
            && \text{by def of $\tilde{r}$}.
        \end{align*}
        Which completes the proof. (The bound established for any state action pair trivially bounds the $\|\cdot\|_\infty$ norm.)
    \end{proof}
\end{lemma}

\begin{lemma}[Soft suboptimality]
    \label{lem:soft_suboptimality}
    Consider an SPI step ($\eta_\theta=\frac{1}{\beta}$), and one has:
    \begin{align*}
        \tilde{f}(\bm{w},\bm{\lambda}) - {f}(\bm{w},\bm{\lambda},\bm{\theta}^{(t)}) \leq \frac{1}{\eta_\theta} \mathbb{E}_{s \sim \tilde{\bm{\mu}}} \Bigg[ \text{D}_\text{KL}\Big( \bm{\pi}^{(t)}(\cdot|s) || \bm{\pi}^{(t+1)}(\cdot|s) \Big) \Bigg].
    \end{align*}
    Refer to \cite{Mei2020} for proof.
\end{lemma}

\begin{lemma}[$Q$-values are Lipschitz in reward and Lagrange parameters for fixed policy]
    \label{lemma:sameQ_is_lipschitz_wrt_w}
    Consider the $Q$ values induced by the same policy on two different rewards in a constrained, Shannon-regularized Markov decision process, we have that, assuming assumption \ref{assumption:bounded_reward_grads} is satisfied, for any $\bm{w}_1,\bm{w_2} \in \text{dom}(\bm{w})$ and any $\bm{\lambda}_1,\bm{\lambda}_2\in\mathcal{B}$, the inequality below holds:
    \begin{align*}
        \| \bm{Q}^{*}_{\tilde{\bm{r_1}}} - \bm{Q}^{*}_{\tilde{\bm{r_2}}} \|_\infty  
        \leq \frac{B_w}{1 - \gamma} \| \bm{w}_1 - \bm{w}_2 \|_2 + \frac{\|\Psi\|}{1 - \gamma} \| \bm{\lambda}_1 - \bm{\lambda}_2 \|_2.
    \end{align*}
    \begin{proof}
        We follow a derivation very close to the one of Lemma B.3 in \cite{Zeng2022}. We show lipschitzness by bounding the gradient with respect to parameters $\bm{\lambda}$ and $\bm{w}$, since the derivation is the same for both parameters, we write $\nabla_\lozenge$ for the gradient with respect to either $\bm{\lambda}$ or $\bm{w}$. We start by writing out the expectation form of the $Q$-value, for some policy $\pi$:
        \begin{align*}
            Q_{\tilde{r}}^\pi(s,a) =
            \mathbb{E}_{\tau \sim \pi}\Bigg[ 
                \sum_{t=0}^{+\infty} \gamma^t (\tilde{r}(s_t,a_t)+H(\bm{\pi}(\cdot|s_t))) 
                \Bigg| s_0 =s, a_0 = a
            \Bigg].
        \end{align*}
        Taking the difference for two different diminished rewards $\tilde{\bm{r}}_1$ and  $\tilde{\bm{r}}_2$ we have:
        \begin{align*}
            |Q_{\tilde{r}_1}^\pi(s,a) - Q_{\tilde{r}_2}^\pi(s,a)|
            &=
            \Bigg|
            \mathbb{E}_{\tau \sim \pi}\Bigg[ 
                \sum_{t=0}^{+\infty} \gamma^t (\tilde{r}_1(s_t,a_t)-\tilde{r}_2(s_t,a_t)\\ &+H(\bm{\pi}(\cdot|s_t))-H(\bm{\pi}(\cdot|s_t))) 
                \Bigg| s_0 =s, a_0 = a
            \Bigg]
            \Bigg|\\
            &\stackrel{(i)}{=}
            \Bigg|
            \mathbb{E}_{\tau \sim \pi}\Bigg[ 
                \sum_{t=0}^{+\infty} \gamma^t (\tilde{r}_1(s_t,a_t)-\tilde{r}_2(s_t,a_t))
                \Bigg| s_0 =s, a_0 = a
            \Bigg]
            \Bigg|\\
            &\stackrel{(ii)}{\leq}
            \mathbb{E}_{\tau \sim \pi}\Bigg[ 
                \sum_{t=0}^{+\infty} \gamma^t 
                \big|
                \tilde{r}_1(s_t,a_t)-\tilde{r}_2(s_t,a_t)
                \big|
                \Bigg| s_0 =s, a_0 = a
            \Bigg]
            \\
            &\stackrel{(iii)}{\leq}
            \mathbb{E}_{\tau \sim \pi}\Bigg[ 
                \sum_{t=0}^{+\infty} \gamma^t 
                B_w \|\bm{w}_1-\bm{w}_2\|_2 \\ &
                + \|\Psi\| \|\bm{\lambda}_1-\bm{\lambda}_2\|_2 
                \Bigg| s_0 =s, a_0 = a
            \Bigg]
            \\
            &\stackrel{(iv)}{=}
                \frac{B_w}{1-\gamma} \|\bm{w}_1-\bm{w}_2\|_2 
                + \frac{\|\Psi\|}{1-\gamma} \|\bm{\lambda}_1-\bm{\lambda}_2\|_2 
            \\
        \end{align*}
        In $(i)$ the regularizers cancel out, then in $(ii)$ we use Jensen's inequality, in $(iii)$ we just plug in the lipschitzness of the reward parametrization (assumption \ref{assumption:bounded_reward_grads} and by definition of the diminished reward). Finally  in $(iv)$ we observe that what is under the expectation is deterministic and that the sum is a geometric series. Since the bound established for any state action pair trivially bounds the $\|\cdot\|_\infty$ norm this completes the proof.
    \end{proof}
\end{lemma}

\begin{lemma}[$\tilde{f}$ is convex]
    \label{lemma:ftilde_is_convex}
    \begin{proof}
        This is a classical convex analysis result that can be found in \cite{boyd2004convex}. Consider $f(\bm{w},\bm{\lambda},\bm{\theta})$ convex in $\bm{z} = [\bm{w},\bm{\lambda}]$, assuming that $\tilde{\bm{\theta}} := \arg \max_{\theta} f(\bm{w},\bm{\lambda},\bm{\theta})$ for fixed $\bm{z}$, and $\tilde{f}(\bm{z}):=f(\bm{w},\bm{\lambda},\tilde{\bm{\theta}})$ Starting with the definition of convexity we have that, for $\gamma\in[0,1]$ and any $\bm{z}_1,\bm{z}_2 \in \text{dom}(\bm{z})$:
        \begin{align*}
            \gamma \tilde{f}(\bm{z}_1) + (1-\gamma) \tilde{f}(\bm{z}_2) 
            & =
            \max_\theta {f}( \gamma \bm{z}_1,\theta) + \max_\theta {f}(  (1-\gamma) \bm{z}_2, \theta ) \\
            & \geq
            \max_\theta \Big( {f}( \gamma \bm{z}_1,\theta)  + {f}(  (1-\gamma) \bm{z}_2, \theta )  \Big)\\
            & \geq
            \max_\theta {f}( \gamma \bm{z}_1 + (1-\gamma) \bm{z}_2, \theta ) \\
            & =
            \max_\theta \tilde{f}( \gamma \bm{z}_1 + (1-\gamma) \bm{z}_2) .
        \end{align*}
        Which verifies by the convexity of $f$, $\tilde{f}$ indeed satisfies the definition of convexity.
    \end{proof}
\end{lemma}